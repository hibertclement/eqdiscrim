The first step in any discrimination study is to know and understand your
data. So here are some analyses of the SiHex catalogue, which we shall use to
train our seismic event discriminator.

\section{The SiHex catalog}
{\bf Short summary of the SiHex project and description of the preferential
location catalogue. Mention the fact that much work was done on homogenizing
the magnitudes to Mw, but that many depths are incorrect due to the 1D velocity
model used. Cite the SiHex papers, the report and the annex to the report.}

The version of the SiHex catalog used for the earthquake part of the study is
the publicly available one (cite source). It contains tectonic earthquakes
within the SiHex zone (i.e. the French economic zone + 20 km). We use all the
earthquakes in the catalog except two: event 640818 which has no associated
depth, and 159405 whose depth is inconsistent with the tectonics of
metropolitain France (125km).  
The catalog of non-tectonic events comes from the research-only catalog, more
precisely from an extraction made by Sophie Merer from the SiHex database.
{\bf At a future date, the analysis here will have to be redone using the SiHex
database that will be housed at EOST, and I'll be able to document exactly the
extraction process.} We only use the events that occur within the SiHex zone,
and remove 14 events whose event-type is non-standard (see standard list
below).  The catalog of stations comes from an extraction that Sophie Lambotte
did on the RéNaSS catalog. The start and end dates correspond to the first and
last arrival time pick available in the database. We use all the available
stations.

The two event catalogs are parsed to create tidy tables with the following
columns :
\begin{description}
\item[ID] A unique integer identifier for each event. It corresponds to the
identifier of the event in the SiHex database.
\item[OriginTime] The UTC origin time of the event.
\item[X] The x-coordinate of the event in a reduced-coordinate system. To
obtain the reduced-coordinate system, the latitude and longitude of each event
is first tranformed into UTM $x$ and $y$ coordinates using the WGS84 elipsoid
and the UTM reference zone number 31. Then the coordinates of all the tectonic
earthquakes in the SiHex zone are scaled using a {\tt StandardScaler} from
{scikit-learn}.
\item[Y] The y-coordinate of the event in the reduced-coordinate system
described above.
\item[Mw] The moment magnitude of the event.
\item[Author] The organization that provided the preferred epicentral location
for the event. The possible values are:
\begin{itemize}
\item LDG = Laboratoire de Detection Géophysique (CEA)
\item continue here
\end{itemize}
\item[LocalHour] The local (i.e. not UTC) hour of the day corresponding to the
origin time. This parameter is a floating point number in the range $[0, 24)$.
\item[LocalWeekday] The local (i.e. not UTC) day of the week corresponding to
the origin time. This parameter is an integer in the range $[1, 7]$ where $1$
corresponds to Monday and 7 corresponds to Sunday.
\item[DistanceStation1] The Euclidean distance in the reduced coordintate
system to the closest station that was running at the time of the event. 
\item[DistanceStation2] The Euclidean distance in the reduced coordintate
system to the second closest station that was running at the time of the event. 
\item[DistanceStation3] The Euclidean distance in the reduced coordintate
system to the second closest station that was running at the time of the event. 
\item[Type] A string containing the type of the event. Possible types are:
\begin{itemize}
\item ke = known earthquake;
\item se = suspected earthquake;
\item km = known mining event;
\item sm = suspected mining event;
\item me = marine event (an explosion that occurred underwater);
\item kr = known rockburst;
\item sr = suspected rockburst;
\item ki = known induced event;
\item si = suspected induced event;
\item uk = unknown;
\end{itemize}
\end{description}
The station catalog is parsed to create a tidy table with the following columns:
\begin{description}
\item[Name] Station name code. Serves as unique identifier for a station.
\item[X] The x-coordinate of the station in the reduced-coordinate system.   
\item[Y] The y-coordinate of the station in the reduced-coordinate system.   
\item[StartTime] The UTC time at which the station started operation.
\item[EndTime] The UTC time at which the station ended operation. Currently
active stations have an end-time in the year 2599.
\end{description}

\section{Data exploration}
The seismic monitoring network in France has evolved enourmously over the SiHex
time period (1962-2009) (Fig.~\ref{fg:stations_by_year}, which leads to a
significant geographical variation of detected earthquakes as a function of
time (Fig.~\ref{fg:events_by_year}). In order to train a discrimination
algorithm on such time-dependent data, we would either need to explicitly take
the time variation into account (for example by training on time-based subsets
of the data) or by regrouping the events by a non-time dependent but physically
meaningful attribute such as the distance to the closest stations at the
time of the event. We have chosen this second alternative, as it will
generalize much better in the future, as the installation of the RESIF-core
stations will again modify the seismic network layout.
Fig~\ref{fg:station_clusters} shows the grouping achieved by k-means clustering
the SiHex earthquakes into three clusters based on the distances to the three
closest stations.

\begin{figure}
\centering
\begin{subfigure}[t]{0.95\textwidth}
	\centering
	\includegraphics[width=\textwidth]{../figures/num_stations_per_year}
	\caption{Number of stations as a functio nof year. This graph was used
	to define four epochs: before 1970 (median number of stations 33),
	1970-1978 (77 stations), 1978-1990 (102 stations), and after 1990 (125
	stations).} 
	\label{fg:stations_by_year}
\end{subfigure}
~
\begin{subfigure}[t]{0.95\textwidth}
	\centering
	\includegraphics[width=\textwidth]{../figures/year_pdf_by_author}
	\caption{Probability density function of earthquake occurrence as a
	function of year. The increase in number of events over time is due to
	the progressive installation of seismic networks.}
	\label{fg:events_by_year}
\end{subfigure}
\caption{Time evolution of the seismic network and the number of recorded
events.}
\label{fg:network_evolution}
\end{figure}

\begin{figure}
\centering
\includegraphics[width=\textwidth]{../figures/clusters_dist_to_3_closest_stations_by_epoch}
\caption{K-means custering of the SiHex earthquakes into three clusters based
on the distances to the three closest stations. Clusters are labeled 0, 1 and
2, and are ordered by increascing combined distance. Maps are shown as a
function of epoch.}
\label{fg:station_clusters}
\end{figure}

We expect that the magnitude of the detected events will also depend on the
distance to the closest available stations at the time of the event. This
effect is shown in Fig.~\ref{fg:mag_by_cluster} by the shift towards low
magnitude events as the distance to the closest stations diminishes.

\begin{figure}
\centering
\begin{subfigure}[t]{\textwidth}
	\centering
	\includegraphics[width=\textwidth]{../figures/mag_pdf_by_station_cluster}
	\caption{Probability density function of $M_W$ for the SiHex
	earthquakes. The right pannel shows the break-down by cluster.
	} 
	\label{fg:hist_mag_by_cluster}
\end{subfigure}
~
\begin{subfigure}[t]{\textwidth}
	\centering
	\includegraphics[width=\textwidth]{../figures/GR_by_station_cluster}
	\caption{Gutenberg-Richter relations for the SiHex earthquakes. The
	right pannel shows the break-down by cluster.}
	\label{fg:GR_by_cluster}
\end{subfigure}
\caption{Dependence of detected magnitudes on the distance to the closest
stations.}
\label{fg:mag_by_cluster}
\end{figure}

Two of the parametric attributes that we might wish to use to distinguish
between anthropogenic and natural events are the local time of day (our
LocalHour parameter) and the local day of the week (our LocalWeekday
parameter). We expect natural earthquakes to be uniformly distributed over
these two attributes, while we expect the anthropogenic events to be more
prevalent during the daylight hours of working days.
Figure~\ref{fg:detectability_hour} shows that the probability density function
of natural earthquakes as a function of local hour of the day is not uniform,
but peaks at night, with a smaller peak at noon-time.  This effect has been
observed before in other regions (cite) and corresponds simply to variations in
the anthropogenic noise that influence the detectability of seismic events. The
detectability variations are smaller for the cluster corresponding to the
shortest distances to the three closest stations. The detectability of natural
earthquakes as a function of day of the week is nearly uniform
(Fig.~\ref{fg:detectability_day}).

\begin{figure}
\centering
\begin{subfigure}[t]{\textwidth}
	\centering
	\includegraphics[width=\textwidth]{../figures/hour_pdf_by_station_cluster}
	\caption{Probability density function of local hour for the SiHex
	earthquakes. The thick horizontal black line indicates the probability
	density value for a uniform distribution.  The right pannel shows the
	break-down by cluster. 
	} 
	\label{fg:detectability_hour}
\end{subfigure}
~
\begin{subfigure}[t]{\textwidth}
	\centering
	\includegraphics[width=\textwidth]{../figures/weekday_pdf_by_station_cluster}
	\caption{Probability density function of local day of the week for the
	SiHex earthquakes.  The thick horizontal black line indicates the
	probability density value for a uniform distribution.  The right pannel
	shows the break-down by cluster.
	} 
	\label{fg:detectability_day}
\end{subfigure}
\caption{Detectability of SiHex earthquakes as a function of local hour and
local day of the week.}
\label{fg:detectability}
\end{figure}

The non-tectonic SiHex catalog is far from complete, as policies concerning the
location and storage of non-tectonic events differ between agencies. The vast
majority of non-tectonic events is discriminated by eye and rejected before
being located. When analysing this catalog, therefore, it is important to bear
in mind that the events we have access to may be statistically biassed.

Fig.~\ref{fg:notecto_day_night} addresses our expectation that non-natural
events should be more prevalent during daylight hours of working days, by
dividing up the non-tectonic SiHex catalog into two subsets: the first contains
the events that occorred during daylight hours of the workweek (defined to be
between 8am and 7pm, Monday to Friday); the second contains all other
events and constitues 23.7\% of the entire non-tectonic catalogue. Given the
unknown statistical bias of the catalogue, we should not try to over-interpret
the relative prevalences of event types. If we limit ourselves to the workday /
other times breakdown of the catalog, we see two main families of event types:
those whose breakdown is far from uniform (i.e.  km, sm, and me) and those
whose breakdown is close to uniform (i.e. ki, si and
uk). Physically speaking, these families are consistent with our expectation:
mining and marine explosions are set off manually, and should only occur during
the working day, while induced events (e.g. from geothermal plants or
hydroelectric dams) can occur at any time. Unknown events seem to be
uniform over time, which might indicate that a certain proportion of them may
actually be small, shallow natural events. The behaviour of the rockbursts
seems to be intermediate between the two families.

\begin{figure}
\centering
\begin{subfigure}[t]{\textwidth}
	\centering
	\includegraphics[width=\textwidth]{../figures/notecto_type_bar_chart}
	\caption{(Left) Bar chart of number of events as a function of non-tectonic
	event type. (Right) Fraction of events of each category that occur
	during the workday, or at other times. The thick black horizontal line
	indicates the expected fraction for a uniform distribution over time.} 
	\label{fg:notecto_bar}
\end{subfigure}
~
\begin{subfigure}[t]{\textwidth}
	\centering
	\includegraphics[width=\textwidth]{../figures/notecto_type_pie_comparison}
	\caption{Pie chart of the type distribution of non-tectonic events
	occurring during the workday or at other times.} 
	\label{fg:notecto_pie}
\end{subfigure}
\caption{Distributions of non-tectonic events according the time of day and day
of the week, as a function of event type.}
\label{fg:notecto_day_night}
\end{figure}

If we analyse the two families and the rockbursts in terms of their local time
of day and day of week distributions, we obtain Figs.~\ref{fg:notecto_hour}
and~\ref{fg:notecto_day}.
